\documentclass[12pt]{article}
\usepackage[utf8]{inputenc}
\usepackage[T1]{fontenc}
\usepackage{amsmath,amsfonts,amssymb}
\usepackage{graphicx}
\usepackage{a4wide}
\title{Reconstructed abstract of the paper ``Spatio-temporal filling of missing points in geophysical data sets''}
%\author{not specified, not necessary here}
\date{}
\begin{document}
\maketitle

\begin{abstract}

Geophysical data are often full of gaps, due to measurement conditions.  In this work, we apply the singular spectrum analysis (SSA) method to fill in gaps. For a multivariate data we use M-SSA takes advantage of spatial and temporal correlations. SSA relies on embedding a time series into a high-dimensional vector space. The time series parts can be reconstructed by linear combinations of principal components. The method`s efficiency was evaluated and improved by cross-validation. Finally, the research demonstrated that SSA is a  promising approach for addressing gaps in geophysical data.



\end{abstract}
\paragraph{Keywords:} singular spectrum analysis, principal components, time series, eigenvalues, periodic eigenvectors, spatial and temporal correlations
 

\paragraph{Highlights:}
\begin{enumerate}
\item Using singular spectrum analysis to fill the gaps in geophysical data.
\item The time series parts can be reconstructed by linear combinations of principal components.
\item SSA is a data-adaptive, nonparametric method based on embedding a time series in a vector space.
\end{enumerate}

\section{Introduction}
This article\cite{kondrashov2006spatio} carries a completely new, flexible and non-parametric method for filling in gaps in geophysical data. Improving the data will simplify the process and increase the quality of research. 
%\begin{figure}
%\includegraphics[scale=0.35]{SVD_derint}
%\caption{A rigorous description of what the reader sees on the plot and the consequences of the shown result}
%\end{figure}

\bibliographystyle{unsrt}
\bibliography{bibliography}
\end{document}